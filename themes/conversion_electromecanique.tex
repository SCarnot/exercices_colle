\chapter{Conversion électromécanique}

\newpage

\section{Moteur synchrone}
 
On considère une machine synchrone cylindrique de longueur $h$ selon $\vec{e}_z$, dont le rotor (en rouge, de rayon $a$) et le stator (en bleu) sont séparés par l'entrefer d'épaisseur $e\ll a$ et sont constitués d'un matériau ferromagnétique linéaire de perméabilité relative $\mu_r$ très grande. 

\begin{figure}[h!]
\centering
		\includegraphics[scale=0.7]{conversion_electromecanique1_1.pdf}
\end{figure}

Deux bobinages, décalés de $\pi/2$, sont enroulés autour du stator et parcourus par les courants $i_1(t)=I_s\cos(\omega t)$ et $i_2(t)=I_s\sin(\omega t)$. Le bobinage autour du rotor est parcouru par un courant d'intensité constante $I_r$. Le rotor peut tourner sans frottement autour de l'axe $z$ et est repéré par l'angle $\theta_r$. Un point $M$ quelconque de l'entrefer est repéré par l'angle $\gamma$.

\begin{enumerate}

	\item Déterminer le champ magnétique $\vec{B}_1(M)$ dans l'entrefer créé par le courant $i_1(t)$, si l'enroulement est uniquement dans le plan $yOz$. Comment modifier cet enroulement de sorte à obtenir un champ magnétique de la forme $\vec{B}_1(M)=K_si_1(t)\cos(\gamma)\vec{e}_r$ ? ($K_s$ est une constante que l'on ne cherchera pas à déterminer).
	
	\item En supposant que l'enroulement parcouru par le courant $i_2(t)$ est identique à celui parcouru par $i_1(t)$ mais décalé de $\pi/2$, montrer que le champ magnétique créé par le bobinage du stator peut s'écrire :
	\begin{align*}
		\vec{B}_s(M)=K_sI_s\cos(\omega t-\gamma)\vec{e}_r
	\end{align*}
	
	\item Expliciter le champ magnétique $\vec{B}_r(M)$ créé par le courant $I_r$ du rotor, si le bobinage est compris dans le plan de normale $\vec{n}$. Comment réaliser un enroulement pour que le champ magnétique puisse s'écrire $\vec{B}_r(M)=K_rI_r\cos(\gamma-\theta_r)\vec{n}$ ?
	
	\item Montrer que l'énergie magnétique contenue dans l'entrefer s'écrit :
	\begin{align*}
		E_m = \frac{\pi eha}{\mu_0}K_sK_rI_sI_r\cos(\omega t-\theta_r) + K
	\end{align*}
	où $K$ est une constante de $\theta_r$.
	
	\item Comment s'exprime le couple électromagnétique $\Gamma_m$ ? Pourquoi appelle t-on cette machine "synchrone" ? En déduire l'angle $\alpha=\omega t-\theta_r$ maximisant le couple.
	
\end{enumerate}

\newpage

\begin{correction}

\begin{enumerate}

	\item On étudie en premier lieu les propriétés du champ magnétique créé par l'enroulement parcouru par le courant $i_1$ :
	\begin{itemize}
		\item le plan $xOy$ est plan d'antisymétrie de la distribution de courant, le champ $\vec{B}$ est donc contenu dans ce plan (on néglige les effets de bord) ;	
		\item le matériau ferromagnétique ayant une perméabilité relative infinie, les lignes de champs sont droites et radiales dans l'entrefer. D'autre part, comme $e\ll a$, on peut considérer le champ magnétique comme constant dans l'entrefer. On a donc pour $a<r<a+e$, $\vec{B}(r, \gamma)=B(\gamma)\vec{e}_r$ ;
		\item le plan $xOz$ est un plan d'antisymétrie donc le champ magnétique est symétrique par rapport à ce plan : $\vec{B}(-\gamma)=\vec{B}(\gamma)$ ;
		\item le plan $yOz$ étant un plan de symétrie, le champ magnétique est antisymétrique par rapport à ce plan, donc $\vec{B}(\pi-\gamma)=-\vec{B}(\gamma)$.
		
	\end{itemize}
	
	On applique le théorème d'Ampère sur le contour représenté en gris (partie supérieure du schéma). 
	
		{\centering
		\includegraphics[scale=0.5]{conversion_electromecanique1_2.pdf}\par }
		
	On a donc :
	\begin{align*}
		\oint_\Gamma\vec{H}\cdot \vec{dl}=\int_A^B\vec{H}\cdot \vec{dl}+\int_B^C\vec{H}\cdot \vec{dl}+\int_C^D\vec{H}\cdot \vec{dl}+\int_D^A\vec{H}\cdot \vec{dl}=i_1(t)
	\end{align*}	 
	Sue les parcours $BC$ et $DA$, l'excitation est nulle : le matériau ayant une perméabilité magnétique infinie, dans le matériau, $H=B/\mu_0\mu_r\longrightarrow0$ le champ magnétique étant le même dans le matériau que dans l'entrefer, par conservation du flux. 
	
	Ainsi :
	\begin{align*}
		-eH(\pi-\gamma)+eH(-\gamma)=-\frac{eB(\pi-\gamma)}{\mu_0}+\frac{eB(\gamma)}{\mu_0}= i_1(t)
	\end{align*}
	En utilisant $\vec{B}(-\gamma)=\vec{B}(\gamma)$ :
	\begin{align*}
		&\gamma\in\left] -\frac{\pi}{2},\frac{\pi}{2}\right[ ,\quad\vec{B}(\gamma)=\frac{\mu_0i_1(t)}{2e}\vec{e}_r\\
		&\gamma\in\left]\frac{\pi}{2},\frac{3\pi}{2}\right[ ,\quad\vec{B}(\gamma)=-\frac{\mu_0i_1(t)}{2e}\vec{e}_r
	\end{align*}
	
	Le champ magnétique est réparti dans l'entrefer comme représenté par la courbe grise en pointillé ci-dessous :
	
		{\centering
		\includegraphics[scale=0.6]{conversion_electromecanique1_3.pdf}\par }
		
	On obtient donc une fonction créneau paire, de valeur moyenne nulle. 
	
	Pour se rapprocher d'un champ variant en $\cos\gamma$, on peut ajouter $N$ enroulements parcourus par $i_1(t)$ décalés d'un angle $\theta_k=\frac{\pi}{N}\left(k-\frac{N-1}{2} \right)$, $k\in\left[0, N-1\right]$, avec un nombre d'enroulements proportionnel à $\propto\cos\theta_k$ (de façon à avoir un champ plus intense pour $\theta_k=0$). Si $M$ est le nombre d'enroulements à $\theta_k=0$, le champ magnétique total créé par le courant $i_1(t)$ s'écrit :
	\begin{align*}
		\vec{B}_1(\gamma)=\sum_{k=0}^{N-1}M\cos(\theta_k)\vec{B}(\gamma-\theta_k)
	\end{align*}
	On peut montrer que cette fonction tend vers $\cos(\gamma)$ lorsque $N\longrightarrow\infty$ :
	\begin{align*}
		\vec{B}_1(\gamma)=K_si_1(t)\cos(\gamma)\vec{e}_r
	\end{align*}
	
	On retrouve sur la courbe verte ci-dessus l'exemple $N=5$, se rapprochant ainsi du $\cos\gamma$.
		
	\item Il s'agit de la même question que la précédente, mais tout décalé de $\pi/2$ : $\gamma\longleftarrow\gamma-\pi/2$ donc :
	\begin{align*}
		\vec{B}_{2}(\gamma)&=K_si_2(t)\cos(\gamma-\pi/2)\vec{e}_r\\
		&=K_sI_s\sin(\gamma)\vec{e}_r
	\end{align*}
	
	Le champ total s'écrit alors :
	\begin{align*}
		\vec{B}_s(\gamma)&=\vec{B}_{1}(\gamma)+\vec{B}_{2}(\gamma)\\
		&=K_sI_s\cos(\omega t)\cos(\gamma)\vec{e}_r+K_sI_s\sin(\omega t)\sin (\gamma)\vec{e}_r\\
		&=K_sI_s\cos(\omega t-\gamma)\vec{e}_r
	\end{align*}
	Il s'agit du champ glissant. 
	
	\item Le raisonnement pour établir le champ magnétique créé par les enroulements du rotor est identique à celui pour trouver le champ du stator, avec un une translation de l'angle de rotation du rotor $\theta_r$ : $\gamma\leftarrow\gamma-\theta_r$. En effet, le fait que le bobinage se situe sur le rotor et non sur le stator ne change rien au théorème d'Ampère. On a donc :
	\begin{align*}
		&\gamma\in\left] -\frac{\pi}{2}-\theta_r,\frac{\pi}{2}-\theta_r\right[ ,\quad\vec{B}_r(\gamma)=\frac{\mu_0i_1(t)}{2e}\vec{e}_r\\
		&\gamma\in\left]\frac{\pi}{2}-\theta_r,\frac{3\pi}{2}-\theta_r\right[ ,\quad\vec{B}_r(\gamma)=-\frac{\mu_0i_1(t)}{2e}\vec{e}_r
	\end{align*}
	
	Si l'on réalise le même enroulement que dans le cas du stator, on obtient, avec le décalage de $\theta_r$ : 
	\begin{align*}
		\vec{B}_r(M)=K_rI_r\cos(\gamma-\theta_r)\vec{n}
	\end{align*}
	
	\item L'énergie magnétique contenue dans l'entrefer s'écrit :
	\begin{align*}
		E_m &= \iiint_V  dv\frac{B^2}{2\mu_0} \\
		&=\int_0^hdz\int_0^{2\pi}d\gamma\int_a^{a+e}rdr\frac{1}{2\mu_0}(B_s^2+B_r^2+2B_sB_r) \\
		&=\frac{eah}{2\mu_0}\int_0^{2\pi}d\gamma \left( K_s^2I_s^2\cos^2(\omega t-\gamma)+K_r^2I_r^2\cos^2(\theta_r-\gamma)+2K_sK_rI_sI_r\cos(\omega t-\gamma)\cos(\theta_r-\gamma)\right) \\
				&=\frac{eah}{2\mu_0}\left(\pi K_s^2I_s^2+\pi K_r^2I_r^2 +\int_0^{2\pi}d\gamma K_sK_rI_sI_r(\cos(\omega t-\theta_r)+\cos(\theta_r+\omega t-2\gamma))\right) \\
				&=\frac{eah}{2\mu_0}\left(\pi K_s^2I_s^2+\pi K_r^2I_r^2 + 2\pi K_sK_rI_sI_r\cos(\omega t-\theta_r)\right) 
	\end{align*}
	
	\item Le couple électromagnétique $\Gamma_m$ s'exprime comme la dérivée de l'énergie magnétique par rapport à la rotation pièce en mouvement, c'est-à-dire le rotor, dont la rotation est repérée par l'angle $\theta_r$ :
	\begin{align*}
		\Gamma_m=&\frac{\partial E_m}{\partial\theta_r} \\
		=& \frac{eah\pi K_sK_rI_sI_r}{\mu_0}\sin(\omega t-\theta_r)
	\end{align*}
	
	La moyenne temporelle du couple est non nulle uniquement si $\omega t =\theta_r$, c'est la condition de synchronisme. Cette machine est un moteur (ou alternateur) uniquement si la vitesse de rotation du rotor est synchronisée avec celle du champ statorique tournant. 
	
	L'angle $\alpha$ représente l'écart entre le champ moyen statorique et le champ moyen rotorique, et est maximal lorsque $\alpha=\pi/2$.
\end{enumerate}

\end{correction}

\newpage

\section{Alternateur synchrone (Centrale PSI 2021)}

On étudie la production d'énergie électrique par une éolienne au moyen d'un générateur utilisant des aimants permanents. Il est constitué d'un stator (en rouge) intérieur cylindrique de diamètre $2a$ et de longueur $L_r$ selon $\vec{e}_z$. Le rotor (en bleu) a un diamètre intérieur noté $2(e+a)$, avec $e\ll a$ l'entrefer du dispositif et est en rotation sans frottements autour de l'axe $\vec{e}_z$, en notant $\theta_r$ sa position angulaire.

\begin{figure}[h!]
\centering
		\includegraphics[scale=0.6]{conversion_electromecanique3_1.pdf}
\end{figure}

Le rotor et le stator sont constitués d'un matériau ferromagnétique doux de perméabilité magnétique relative $\mu_r$ supposée infinie. Un point $M$ quelconque de l'entrefer est repéré par l'angle $\gamma$.

\begin{enumerate}

	\item On enroule autour du stator un câble parcouru par un courant électrique d'intensité $i_1$. Déterminer le champ magnétique $\vec{B}_1$ créé par cet enroulement.
	
	\item On enroule maintenant un grand nombre de spires, toutes parcourues par $i_1$, dans différents plans et on admet qu'une répartition adéquate permet d'obtenir un champ magnétique statorique dans l'entrefer qui varie sinusoïdalement avec l'angle $\gamma$ selon :
	\begin{align*}
		\vec{B}_1=\frac{N\mu_0i_1}{2e}\cos(\gamma)\vec{e}_r
	\end{align*}
	Proposer qualitativement comment doit se faire la répartition des enroulements pour arriver à ce champ magnétique.
	
	\item On utilise un enroulement statorique identique au précédent, mais décalé de $\pi/2$, et parcouru par $i_2$. Expliciter l'expression $\vec{B}_2$ du champ magnétique créé par cet enroulement.
	
	\item Les courants $i_1$ et $i_2$ sont supposé désormais sinusoïdaux : $i_1(t)=I_0\cos(\omega t)$ et $i_2(t)=I_0\sin(\omega t)$. Montrer que le champ magnétique total créé par les enroulements statoriques peut s'écrire :
	\begin{align*}
		\vec{B}_s=\frac{N\mu_0i_1}{2e}\cos(\omega t-\gamma)\vec{e}_r
	\end{align*}
	\item On admet que le rotor produit, au moyen d'aimants permanents, un champ magnétique dans l'entrefer qu'on considérera comme solidaire du rotor s'écrivant sous la forme : $\vec{B}_r=B_r\cos(\gamma-\theta_r)\vec{e}_r$. Déterminer l'expression de l'énergie magnétique totale dans l'entrefer $E_m$.
	
	\item En déduire l'expression du couple $\Gamma$ exercé sur le rotor.
	
	\item On note $\alpha=\omega t-\theta_r$. A quelle condition la moyenne temporelle du couple $<\Gamma>$ est-elle non nulle ? Tracer $<\Gamma>$ en fonction de $\alpha$, préciser dans quel régime doit-on se trouver pour être en fonctionnement alternateur stable. 

\end{enumerate}

\newpage

\section{Moteur à courant continu}

On considère une machine à courant continu, dont le stator et le rotor, de longueur $L$ suivant l'axe $y$, sont constitués d'un matériau ferromagnétique doux de perméabilité relative $\mu_r$ infinie. 

Le rotor est un cylindre de rayon $a$ et de longueur $L$ pouvant tourner librement autour de l'axe $y$. $N$ fils parcourus par une intensité $I_r$, parallèles à l'axe $y$, sont enroulés autour du rotor. Tous les enroulements sont en série et un système de collecteur permet que le courant $I_r$ soit dirigé suivant $+\vec{e}_y$ pour $z>0$, et suivant $-\vec{e}_y$ pour $z<0$.

Le stator est un parallélépipède évidé de sorte à accueillir le rotor en son sein,  entouré de $N$ enroulements ($N/2$ sur les parties supérieures et inférieures) contenus dans des plans parallèles à $(xOy)$ parcourus par un courant continu $I_s$. La distance $e\ll a$ entre le rotor et le stator est appelée entrefer.

\begin{figure}[h!]
\centering
		\includegraphics[scale=0.5]{conversion_electromecanique2_1.pdf}
\end{figure}

\begin{enumerate}

	\item  A l'aide des symétries du problème et des propriétés du matériau ferromagnétique, tracer l'allure des lignes du champ magnétique $\vec{B}_s$ créé par les enroulements du stator. En déduire que le champ magnétique créé par les enroulements du stator dans l'entrefer peut s'écrire sous la forme :
	\begin{align*}
		\vec{B}_s(\theta_r)&=B_0\vec{e}_r, \quad \theta_r\in [0,\pi] \\
		&=-B_0\vec{e}_r, \quad \theta_r\in [-\pi,0] 
	\end{align*}
	et expliciter l'expression de $B_0$. On rappelle que les lignes de champ sont orthogonales à l'interface dans l'entrefer. 
	
	\item On isole l'enroulement repéré par l'angle $\theta_r$ (en rouge sur le schéma). Quelle force s'applique sur lui ? En déduire le couple $\Gamma_1$ dû à cet enroulement qui s'exerce sur le rotor.
	
	\item En notant $n_0=N/2\pi$ la densité radiale de spires, montrer que le couple total exercé sur le rotor s'écrit $\Gamma = 2\Phi_0 I_r$, où $\Phi_0$ est le flux de $\vec{B}_s$  à travers une surface que l'on explicitera.
	
	\item Montrer que le flux de $\vec{B}_s$ à travers l'enroulement repéré par $\theta_r$ peut s'écrire sous la forme $\phi = f(\theta_r)\Phi_0$, où $f(\theta_r)$ est une fonction décroissante comprise entre -1 et 1. Le rotor tournant à la vitesse $\Omega$, quelle est la force électromotrice $e_1$ créée dans l'enroulement ?
	
	\item En déduire que la force électromotrice totale $e$ créée par l'ensemble des enroulements en série s'écrit :  
	\begin{align*}
		e=2\Phi_0\Omega
	\end{align*}
	Commenter les expressions trouvées. 
	
\end{enumerate}

\newpage

\begin{correction}

\begin{enumerate}

	\item On ne tient compte ici que des enroulements du stator. 
	
	\begin{itemize}
	
		\item Le plan $xOz$ étant un plan de d'antisymétrie de la distribution de courant, $\vec{B}_s$ est contenu dans ce plan. 
		
		\item Le plan $xOy$ étant un plan de de de symétrie de la distribution de courant, $\vec{B}_s$ est antisymétrique par rapport à ce plan : $\vec{B}_s(\pi-\theta_r)=-\vec{B}_s$.
		
		\item La perméabilité du matériau étant infini, les lignes de champ sont radiales dans l'entrefer ; de plus le matériau canalise les lignes de champ.
		
	\end{itemize}	 
	
	En conséquence, les lignes de champ ont l'allure suivante :	
	
	{\centering
		\includegraphics[scale=0.5]{conversion_electromecanique2_2.pdf}\par }
		
	On en déduit donc que $\vec{B}_s$ est radial, dirigé suivant $+\vec{e}_r$ si $\theta_r\in [0,\pi]$ et dirigé suivant dirigé suivant $\vec{e}_r$ si $\theta_r\in [-\pi,0]$. 
	
	On applique ensuite le théorème d'Ampère sur une ligne de champ (quelconque), dont on note $L_{fer}$ (resp. $H_{fer}$) la longueur (resp. l'excitation) dans le matériau ferro :
	\begin{align*}
		L_{fer}H_{fer} + 2eH_e=NI_s
	\end{align*}
	Comme l'excitation magnétique dans le matériau est nulle (car $\mu_r\rightarrow\infty$), on a donc, dans l'entrefer :
	\begin{align*}
		B_0 = \frac{\mu_0NI_s}{2e}
	\end{align*}
	
	\item Il s'agit de la force de Laplace $d\vec{F} = I\vec{dl}\wedge\vec{B}$. Appliqué sur le fil repéré par $\theta_r$ (en $z>0$) de l'enroulement en rouge, on obtient :
	\begin{align*}
		\vec{F}=\int_0^L I_rdy\vec{e}_y\wedge \vec{B_s}=I_rLB_0\vec{e}_{\theta_r}
	\end{align*}
	Sur le fil en $\theta_r+\pi$, pour $z<0$ :
	\begin{align*}
		\vec{F}=\int_0^L I_r(-dy)\vec{e}_y\wedge \vec{B_s}=-I_rLB_0\vec{e}_{\theta_r}
	\end{align*}
	Cet enroulement exerce un couple résultant sur l'axe $y$ s'écrivant :
	\begin{align*}
		\Gamma_1=2aLI_rB_0
	\end{align*}
	
	\item Le couple exercée par les $dN=n_0d\theta$ enroulements situés entre $\theta$ et $\theta+d\theta$ est la somme des couples exercée par les enroulements individuels :
	\begin{align*}
		\Gamma &=\int_{-\pi/2}^{\pi/2}d\theta \Gamma_1 n_0d\theta \\
		&=2\pi aLI_rB_0 \\
		&=2\Phi_0B_0
	\end{align*}
	$B_0$ étant la norme du champ $\vec{B}_s$ au niveau de l'entrefer, en y étant orthogonal à l'interface, et la demi-surface du cylindre du rotor est $S=\pi a L$, on en déduit que $\Phi_0$ est le flux du champ magnétique dans l'entrefer pour $z>0$ (ou $z<0$).
	
	\item Regardons le flux passant à travers les enroulements situés, à un instant $t$, aux positions $\theta_r=0$, $\pi/4$ et $\pi/2$ : 
	
	{\centering
		\includegraphics[scale=0.5]{conversion_electromecanique2_3.pdf}\par }
		
	Pour $\theta_r=0$, l'enroulement est dans le plan $(xOy)$, qui un plan de symétrie de distribution de courant. Le champ magnétique est donc orthogonal à ce plan. Par conservation du flux, le flux du champ magnétique $\Phi_0$ calculé à la question précédente passe nécessairement intégralement par la surface délimitée, donc $\phi(\theta_r=0)=\Phi_0$.
	
	Pour $\theta_r=\pi_2$, l'enroulement est confondu avec le plan ($yOz$), qui est plan d'antisymétrie de la distribution de courant, donc $\vec{B}_s$ appartient à ce plan, son flux est donc nul : $\phi(\theta_r=\pi/2)=0$.
	
	Pour $\theta_r=\pi_2$, il est impossible de déterminer analytiquement le flux, mais on se trouve dans une situation intermédiaire aux deux précédentes, avec un flux à travers l'enroulement non nul, mais inférieur à $\Phi_0$ : $0<\phi(\theta_r=\pi/4)<\Phi_0$.
	
	Enfin par symétrie, $\phi(\theta_r=\pi)=-\Phi_0$.
	
	On en déduit que la fonction $f$ définie par $\phi = f(\theta_r)\Phi_0$ est comprise entre $f(0)=1$ et $f(\pi)=-1$ (donc décroissante de $\theta_r$).
	
	Lorsque l'enroulement tourne à la vitesse $\Omega$, la force électromotrice induite par le champ $\vec{B}_s$ est :
	\begin{align*}
		e_1=-\frac{d\phi}{dt}=-\Phi_0 f'(\theta_r)\dot{\theta_r}=-\Phi_0 f'(\theta_r)\Omega
	\end{align*}
	
	\item Tous les enroulements étant en série, la force électromotrice crée par les $dN=n_0d\theta$ enroulements situés entre $\theta$ et $\theta+d\theta$ est la somme des forces électromotrices exercée par les enroulements individuels :
	\begin{align*}
		e=&\int_0^\pi e_1 n_0d\theta \\
		=&-\Phi_0 \Omega\int_0^\pi f'(\theta) n_0d\theta \\
		=&-\Phi_0 \Omega\left[f(\theta) \right]_{0}^{\pi} \\
		=&2\Phi_0 \Omega
	\end{align*}
	
	\item On retrouve les expressions du cours, où le couple est proportionnel à l'intensité circulant dans le bobinage du rotor et la force contre électromotrice $-e$ est proportionnelle à la vitesse de rotation. Cet exercice nous permet de rendre compte de la constante de proportionnalité, $2\Phi_0$, correspondant au double du flux du champ magnétique statorique à travers les enroulements du rotor. 
\end{enumerate}

\end{correction}