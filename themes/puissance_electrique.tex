\chapter{Puissance électrique en régime sinuoïdal}

\newpage

\section{Puissance absorbée par une installation}

Une installation électrique, constituée de nombreuses machines tournantes, équivalente à une résistance et une bobine en série $(R_1,L_1)$, est alimentée par la tension du réseau de valeur efficace $U=220$V à la fréquence $f=50$Hz.

\begin{center}
	\includegraphics[scale=1.2]{puissance_electrique1_1.pdf}
\end{center}

\begin{enumerate}

    \item \textit{Schéma (a)} : L'installation est de type inductif, qui consomme une puissance $P_1=2,0$kW et est parcourue par un courant d'intensité efficace $I_1=18,2$A. En déduire les valeurs de $R_1$ et l'inductance $L$. Calculer le facteur de puissance $\cos(\varphi_1)$.

    \item \textit{Schéma (b)} : Un radiateur est ajouté à l'installation, équivalent à une résistance $R_2$ en parallèle consommant une puissance $P_2=1,0$kW. Calculer les valeurs efficaces $I$, $I_1$, et $I_2$, la puissance $P$ et le facteur de puissance $\cos(\varphi_2)$.

    \item \textit{Schéma (c)} : on veut rendre égal à 1 le facteur de puissance $\cos(\varphi)$ de l'installation. Pourquoi le distributeur d'électricité impose cette condition ? Quelle valeur de capacité $C$ doit-on placer en parallèle ?

\end{enumerate}

\newpage

\begin{correction}

    \begin{enumerate}

        \item On définit l'origine des phases par la tension : $u=\sqrt{2}U\cos(\omega t)$, le courant $i_1$ étant en déphasage de $\varphi$ par rapport à $u$ : $i_1(t)=\sqrt{2}I_1\cos(\omega t\varphi_1)$.
        On sait que la puissance est, en moyenne, uniquement dissipée dans la résistance : $P=R_1I_1^2$. On en déduit donc $\boxed{R_1=6,04\Omega}$.

        D'autre part, $\underline{i_1}=(R_1+jL_1\omega)\underline{u}$, donc, en prenant le module : 
        \begin{align*}
            \boxed{L_1=\frac{1}{\omega}\sqrt{\frac{U^2}{I_1^2}-R_1^2}=33\mathrm{mH}}
        \end{align*}

        Pour le facteur de puissance, on utilise simplement la formule $P_1=UI_1\cos(\varphi_1)$, on trouve $\boxed{\cos(\varphi_1)=0,50}$ (donc $\varphi_1=\pi/3$).

        \item Dans la branche où circule $i_2$, il n'y a pas de déphasage comme il n'y a ni inductance, ni capacité. On a donc $\boxed{I_2=P_2/U=4,55\mathrm{A}}$.

        Pour trouver les autres grandeurs, on s'appuie sur un diagramme de Fresnel, qui simplifie l'analyse : 

        {\centering
		\includegraphics[scale=1.6]{puissance_electrique1_2.pdf}\par }

        L'intensité efficace $I_1=18,2$A reste inchangée, étant donné qu'elle est alimentée par la même tension. Avec le diagramme de Fresnel, on en déduit l'intensité $I$ :
        \begin{align*}
            \boxed{I=\sqrt{(I_1\cos(\varphi_1)+I_2)^2+I_1^2\sin(\varphi_1)^2}=20,9\mathrm{A}}
        \end{align*}

        Avec le diagramme de Fresnel, on trouve le facteur de puissance : 
        \begin{align*}
            \boxed{\cos(\varphi_2)=\frac{I_1\cos(\varphi_1)+I_2}{I}=0,65}
        \end{align*}

        La puissance est tout simplement la somme des deux puissances dissipée dans les résistances : $P=3,0$kW.

        \item Le distribiteur impose cette condition afin de minimiser l'intensité efficace dans le réseau à puissance fixée. En effet, les lignes électriques ayant une résistance faible mais non nulle, à puissance consommée chez l'utilisateur, un facteur de puissance faible engendrera une intensité élevée dans le réseau et une dissipation d'autant plus importante. 

        Pour mettre le facteur de puissance égal à 1, il faut que le courant $I_3$ "remonte" le courant total de sorte à ce qu'il ait une partie imaginaire nulle (cf schéma ci-dessus). 
        On doit donc avoir : 
        \begin{align*}
            I_3=I_1\sin(\varphi_1)=UC\omega
        \end{align*}       
        On trouve donc : 
        \begin{align*}
            \boxed{C = \frac{I_1\sin(\varphi_1)}{U\omega}=2,3\times10^{-4}\mathrm{F}}
        \end{align*}  
    \end{enumerate}

\end{correction}