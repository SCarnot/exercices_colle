\chapter{Magnétostatique}

\newpage

\section{Champ magnétique dans un cylindre parcouru par un courant orthoradial}

On considère un cylindre conducteur de rayon $a$ et de longueur $L\gg a$ selon l'axe $O_z$, dans lequel circule une densité volumique de courant $\vec{j}(r)=j_0\frac{r}{a}\vec{e}_\theta$.

\begin{enumerate}

	\item A l'aide des symétries et invariances, expliciter la dépendance spatiale et la direction du champ magnétique.
	\item Déterminer l'expression du champ magnétique $\vec{B}(r)$ en fonction de la valeur du champ magnétique en $r=0$.
	\item Quel est l'expression du champ magnétique $\vec{B}(r)$ si on impose un champ extérieur $\vec{B}_{ext}$ de sorte à ce que  $\vec{B}(r=a)=\vec{0}$ ? Quelle est alors la valeur de $\vec{B}_{ext}$ ?

\end{enumerate}

\newpage

\begin{correction}

\begin{enumerate}

	\item Invariance : le champ ne dépend que de $r$. Symétrie : le plan $(\vec{e}_r ;\vec{e}_\theta)$ est plan de symétrie de la distribution de courant donc $\vec{B}$ est suivant $\vec{e}_z$.
	
	\item On calcule la circulation de $\vec{B}$ sur le contour $\Gamma$ :
	
	{\centering
		\includegraphics[scale=0.25]{magnetostatique1_1.pdf}\par }
		
	\begin{align*}
		\oint_\Gamma d\vec{l}\cdot\vec{B}&=\int_B^Cdz\times B(r)+\int_D^Adz\times B(0) \\
		&=-hB(r)+hB(0)
	\end{align*}	 
	D'après le théorème d'Ampère, pour $r<a$ :
	\begin{align}
		-hB(r)+hB(0)&=\mu_0j_0h\frac{r^2}{2a}
	\end{align}
	Donc :
	\begin{align*}
		B(r) = B(0)-\mu_0j_0\frac{r^2}{2a}
	\end{align*}
	
	Pour $r>a$ :
	\begin{align*}
		B(r) = B(0)-\mu_0j_0\frac{a}{2}
	\end{align*}	


	{\centering
		\includegraphics[scale=0.25]{magnetostatique1_2.pdf}\par }
	
	\item On ajoute un champ magnétique extérieur $\vec{B}_{ext}$, nécessairement selon $\vec{e}_z$ :
	\begin{align*}
		B(r) = B(0)-\mu_0j_0\frac{r^2}{2a} + B_{ext}
	\end{align*}
	Si $B(r=a)=0$ alors $B_{ext}=-B(0)+\mu_0j_0\frac{a}{2}$ et alors :
	\begin{align*}
		B(r) = \frac{\mu_0j_0}{2a}\left( a^2-r^2\right) 
	\end{align*}


\end{enumerate}

\end{correction}